%=== CHAPTER ONE (1) ===
%=== INTRODUCTION ===

\chapter{Introduction}
\section{About}
We would be investigating the security of N-th degree Truncated polynomial Ring Units which is also known as NTRU for short.
NTRU is an open-sourced public-key cryptosystem which allows two remote parties to communicate in a secure way without sharing a secret key. 
NTRU is required to be secure against the future quantum computer to minimise information intersections and attacks where .
In this project, we will analyze the security of NTRU and delve on the characteristics of lattice-based cryptosystem.
Furthermore, we will study and improve the current existing attacks on NTRU and attempt to develop new attacks against NTRU. 
 The program is available at 
\url{https://github.com/charutomo/Security-Evaluation-of-NTRU-Public-Key-Cryptosystem}.
\textbf{Keywords:} NTRU, open-sourced, public-key cryptosystem, quantum computer
\newline
\par\noindent\rule{\textwidth}{0.4pt}

\section{Background Information}
NTRU was introduced by Hoffstein J., Pipher J. and Silverman J.H. in the year 1996 and was patented the following year by their company, NTRU Cryptosystems Inc, alongside with Lieman D..
\newline
\par\noindent\rule{\textwidth}{0.4pt}

\section{Deliverables}
The expected deliverables for the project is an overall analysis report of the existing attacks calibrating its advantages and suggestive improvement and an attempt on python implementation of a new attack against NTRU. 
%=== END OF CHAPTER ONE ===
\newpage