\chapter{NTRU}
\section{Behind the math}
In this section, we evaluate the Mathematical aspect of NTRU particularly lattice-based cryptography and its characteristics.
This is crucial in understanding the crux of how NTRU works and what can be the potential improvements be made.

\subsection{Lattice-based Cryptography}

\subsection{Characteristics of NTRU}

\par\noindent\rule{\textwidth}{0.4pt}

\section{Existing Attacks}
Now, we would study the current existing attacks such as brute force attack and meet in the middle attack. 
By understanding each attack, it can inspire other forms of attack which can be more targeted to the system.

\subsection{Brute-force attack}
Brute-force is one of the most common attacks for cryptosystems. 
However, it usually can take a longer period to be able to fully attack the system.
Due to large runtime complexity, it may take many permutation to be carried out before determining the actual plaintext.

\subsection{Meet-in-the- middle attack}
Meet-in-the-middle attack, also known as MITM, is a form of cryptanalysis to observe the patterns from both ends until the middle part of the ciphertext.
This attack can break the system in faster and more elegant manner as compared to brute-force as it reduces the number of permutations to discover the secret keys of NTRU.

\subsection{Multiple transmission attacks}

\subsection{Lattice-based attacks}

\par\noindent\rule{\textwidth}{0.4pt}

\section{Proposed new attack}

\subsection{Targeted Attack}

\textbf{Keywords:} 
\par\noindent\rule{\textwidth}{0.4pt}
%=== END OF CHAPTER THREE ===
\newpage